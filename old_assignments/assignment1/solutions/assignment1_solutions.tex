% Template LaTeX source file for homework problem solutions.
% Alan T. Sherman (9/9/98)
% Updated: Greg King (2014)

% Running LaTeX
%
% Name this file FOO.tex
% latex FOO
% latex FOO   
%    (You have to run latex twice to get the cross references correct.
%     Running latex creates a file FOO.dvi 
%     You can view dvi files with the program xdvi )
% xdvi FOO.dvi &
%
% lpr -d FOO.dvi
%    (To print the dvi file.   Be sure to use the "-d" print option,
%     and be sure your printer can handle dvi files (not all printers can).
%     Do NOT print with "lpr FOO.dvi", which will print tens of pages
%     of unreadable dvi source code. Printing a postscript (ps) file
%     is usually more reliable, as explained below.)
%
% dvips FOO.dvi
%    (To create a postscript file named FOO.ps 
%     which you can view with the program ghostview )
% ghostview FOO.ps &
% lpr FOO.ps
%    (To print the ps file.)

%%%%%%%%%%%%%%%%%%%%%%%%%%%%%%%%%%%%%%%%%%%%%%%%%%%%%%%%%%%%%%%%%%%%%%

\RequirePackage{amsmath}
\RequirePackage{tikz}
\usetikzlibrary{calc}
\RequirePackage{tkz-euclide}
\usetkzobj{all}

\documentstyle[12pt]{article}
% Set the margins
%
\setlength{\textheight}{8.5in}
\setlength{\headheight}{.25in}
\setlength{\headsep}{.25in}
\setlength{\topmargin}{0in}
\setlength{\textwidth}{6.5in}
\setlength{\oddsidemargin}{0in}
\setlength{\evensidemargin}{0in}


%%%%%%%%%%%%%%%%%%%%%%%%%%%%%%%%%%%%%%%%%%%%%%%%%%%%%%%%%%%%%%%%%%%%%%%
% Macros

% Math Macros.  It would be better to use the AMS LaTeX package,
% including the Bbb fonts, but I'm showing how to get by with the most
% primitive version of LaTeX.  I follow the naming convention to begin
% user-defined macro and variable names with the prefix "my" to make it
% easier to distiguish user-defined macros from LaTeX commands.
%
\newcommand{\myN}{\hbox{N\hspace*{-.9em}I\hspace*{.4em}}}
\newcommand{\myZ}{\hbox{Z}^+}
\newcommand{\myR}{\hbox{R}}

\newcommand{\myfunction}[3]
{${#1} : {#2} \rightarrow {#3}$ }

\newcommand{\myzrfunction}[1]
{\myfunction{#1}{{\myZ}}{{\myR}}}


% Formating Macros
%

\newcommand{\myheader}[4]
{\vspace*{-0.5in}
\noindent
{#1} \hfill {#3}

\noindent
{#2} \hfill {#4}

\noindent
\rule[8pt]{\textwidth}{1pt}

\vspace{1ex} 
}  % end \myheader 

\newcommand{\myalgsheader}[0]
{\myheader{Stanford University, Department of Computer Science}
{Computer Science 224D}{Spring 2016}{Section 1}}

% Running head (goes at top of each page, beginning with page 2.
% Must precede by \pagestyle{myheadings}.
\newcommand{\myrunninghead}[2]
{\markright{{\it {#1}, {#2}}}}

\newcommand{\myrunningalgshead}[2]
{\myrunninghead{Computer Science 224D}{{#1}}}

\newcommand{\myrunninghwhead}[2]
{\myrunningalgshead{Solution to Assignment {#1}, Problem {#2}}}

\newcommand{\mytitle}[1]
{\begin{center}
{\large {\bf {#1}}}
\end{center}}

\newcommand{\myhwtitle}[3]
{\begin{center}
{\large {\bf Solution to Assignment {#1}, Problem {#2}}}\\
\medskip 
{\it {#3}} % Name goes here
\end{center}}

\newcommand{\mysection}[1]
{\noindent {\bf {#1}}}

%%%%%% Begin document with header and title %%%%%%%%%%%%%%%%%%%%%%%%%
\begin{document}

\myalgsheader

\pagestyle{plain}
\setcounter{page}{1}
\myhwtitle{1}{1}{Gregory King}

\bigskip
%Assn#1:
%1.29, 1.42, 1.43
%%%%% Begin Solution %%%%%%%%%%%%%%%%%%%%%%%%%%%%%%%%%%%%%%%%%%%%%%%%%%%

\noindent {\bf Softmax} Prove that softmax (denoted $\textrm{softmax}({\bf x})$)
is invariant to a constant offset in the input, that is, for any input vector
$\bf x$ and any constant $c$,
\begin{equation}
\textrm{softmax}({\bf x}) = \textrm{softmax}({\bf x} + c),
\end{equation}
where ${\bf x} + c$ means adding the constant $c$ to every dimension of ${\bf x}$.

{\bf Note}: In practice, we make use of this property and choose $c = -\textrm{max}_{i}({\bf x_i})$, when computing $\textrm{softmax}$ probabilities for numerical stability (i.e. subtracting the maximum element from all elements of ${\bf x}$).
% A spelunker is surveying a cave. She follows a passage 180~m straight
% west, then 210~m in a direction $45^{\circ}$ east of south, and then 280~m at
% $30^{\circ}$ east of north. After a fourth unmeasured displacement, she finds
% herself back where she started. Use a scale drawing to determine the magnitude
% and direction of the fourth diplacement.

\noindent\rule{\textwidth}{0.4pt}

\noindent Starting from the definition of softmax,
\begin{equation}
\textrm{softmax}({\bf x})_{i} = \frac{e^{x_{i}}}{\sum_{j=1}{e^{x_{j}}}}
\end{equation}
it follows, the linear shift, ${\bf x} + c$, yields the following (for the $i$ element):
\begin{align}
\textrm{softmax}({\bf x} + c)_{i} & = \frac{e^{(x_{i} + c)}}{\sum_{j=1}{e^{(x_{j} + c)}}}\\
                                                 & = \frac{e^{c}}{e^{c}}\frac{e^{(x_{i})}}{\sum_{j=1}{e^{(x_{j})}}}\\
                                                 & = \textrm{softmax}({\bf x})_{i},\\
\end{align}
since this is true for an arbitary $i$, it follows immediately that the linear shift has no affect on the softmax probabilities.


%\sum_{j=1}{\exp{x_{j}}

% The vector addition diagram is drawn in Figure~\ref{solution figure:question 1.29}.
% Careful measurement gives $\vec{D}$ is around 144~m, $41^{\circ}$ south of west (scale is 1 grid point:50 m).
% \begin{figure}[!h!b]
% \begin{center}
% \begin{tikzpicture}
%   \draw[step=.5cm,gray,very thin](-2.,-2.) grid (2., 2.);
%   \draw[->,black,thick] (-2.,0) -- (2.,0) coordinate (x axis);
%   \draw[->,black,thick] (0,-2.) -- (0,2.) coordinate (y axis);
%   \coordinate (A) at (0,0);%start at orgin
%   \coordinate (B) at (-1.8,0);%go west 180 m
%   \coordinate (C) at (-0.315,-1.485);% go east 210 cos (315*) = 148.5m, go south 148.5m
%   \coordinate (D) at (1.085, 0.94);%go north 280 sin (60*) = 242.5 m, go east 280 sin(60*) = 140m
%   \draw[->,black,very thick] (A) -- node[above=2pt] {$\vec{A}$} (B);
%   \draw[->,black,very thick] (B) -- node[left=2pt] {$\vec{B}$}  (C);
%   \draw[->,black,very thick] (C) -- node[right=2pt] {$\vec{C}$} (D);
%   \draw[->,black,dashed, very thick] (D) -- node[above=2pt] {$\vec{D}$} (A);
% \end{tikzpicture}
% \end{center}\caption{Vector Diagram for (1.29)\label{solution figure:question 1.29}}
% \end{figure}

%% First page break
%%   Note: "myheadings" is a LaTeX keyword; it is not a user-defined entity.
%%
\clearpage
\pagestyle{myheadings}
\myrunninghwhead{1}{2 (Neural Networks)}

\myhwtitle{1}{2(a)}{Gregory King}

\bigskip

\noindent Derive the gradients of the sigmoid function and show that it can be rewritten as a function of the 
function value (i.e. in some expression where only $\sigma(x)$, but not $x$, is present). Assume that the input
$x$ is a scalar for this question.

\noindent\rule{\textwidth}{0.4pt}
 
\noindent alskdfasd.
 
\clearpage
\pagestyle{myheadings}

\myhwtitle{1}{2(b)}{Gregory King}
\bigskip

\noindent Derive the gradient with regard to the inputs of a softmax function when cross entropy loss is used for
evaluation, i.e. find the gradients with respect to the softmax input vector $\theta$, when the prediction is
made by $\hat{y} = \textrm{softmax}(\theta)$. Remember the cross entropy function is
\begin{equation}
\textrm{CE}(y,\hat{y}) = -\sum_{i}{y_{i}\log{\hat{y_{i}}}}
\end{equation}
where $y$ is the one-hot label vector, and $\hat{y}$ is the predicted probability vector for all classes. \vspace{2mm}\\
\noindent {\bf Hint}: you might want to consider the fact many elements of $y$ are zeros, and assume that only the $k$-th dimension
of $y$ is one.

\clearpage

\myhwtitle{1}{2(c)}{Gregory King}
\bigskip

\noindent Derive the gradients with respect to the inputs x to an one-hidden-layer neural network (that is, find
$\partial{J}/\partial{\bf x}$ where $J$ is the cost function for the neural network). The neural network employs sigmoid activation
function for the hidden layer, and softmax for the output layer. Assume the one-hot label vector is $y$,
and cross entropy cost is used. (feel free to use $\sigma^{\prime}(x)$ as the shorthand for sigmoid gradient, and feel free
to define any variables whenever you see fit). \vspace{5mm}\\
\def\layersep{2.5cm}
\begin{center}
\begin{tikzpicture}[shorten >=1pt,->,draw=black!50, node distance=\layersep]
    \tikzstyle{every pin edge}=[<-,shorten <=1pt]
    \tikzstyle{neuron}=[circle,fill=black!25,minimum size=17pt,inner sep=0pt]
    \tikzstyle{input neuron}=[neuron, fill=green!50];
    \tikzstyle{output neuron}=[neuron, fill=red!50];
    \tikzstyle{hidden neuron}=[neuron, fill=blue!50];
    \tikzstyle{annot} = [text width=4em, text centered]

    % Draw the input layer nodes
    \foreach \name / \y in {1,...,4}
    % This is the same as writing \foreach \name / \y in {1/1,2/2,3/3,4/4}
        \node[input neuron, pin=left:$x_{\y}$] (I-\name) at (0,-\y) {};

    % Draw the hidden layer nodes
    \foreach \name / \y in {1,...,5}
        \path[yshift=0.5cm]
            node[hidden neuron] (H-\name) at (\layersep,-\y cm) {};

    % Draw the output layer nodes
    \foreach \name / \y in {1,...,6}
        \path[yshift=0.75cm]
            node[output neuron] (O-\name) at (\layersep+\layersep,-\y cm) {};
%    % Draw the output layer node
%    \node[output neuron,pin={[pin edge={->}]right:Output}, right of=H-3] (O) {};

    % Connect every node in the input layer with every node in the
    % hidden layer.
    \foreach \source in {1,...,4}
        \foreach \dest in {1,...,5}
            \path (I-\source) edge (H-\dest);

%    % Connect every node in the hidden layer with the output layer
    \foreach \source in {1,...,5}
       \foreach \dest in {1,...,6}
        \path (H-\source) edge (O-\dest);

    % Annotate the layers
    \node[annot,above of=H-1, node distance=1cm] (hl) {$h$};
    \node[annot,left of=hl] {$x$};
    \node[annot,right of=hl] {$\hat{y}$};
\end{tikzpicture}
\end{center}
\noindent Recall that the forward propagation is as follows:
\begin{align}
{\bf h} & = \textrm{sigmoid}({\bf x\textrm{W}_{1} + b_{1}}) \\
{\bf \hat{y}} & = \textrm{softmax}({\bf x\textrm{W}_{2} + b_{2}})
\end{align}
\noindent {\bf Note}: here we're assuming that the input vector (and thus the hidden variables and output probabilities)
is a row vector to be consistent with the programming part of the assignment. When we apply the sigmoid function to
a vector, we are applying it to each of the elements of the vector. $\bf{\textrm{W}}_i$ and $\bf{b}_{i}$ ($i\in\{1,2\}$) are
the weights and biases, respectively of the two layers.
\clearpage

\myhwtitle{1}{2(d)}{Gregory King}
\bigskip
\noindent How many parameters are there in this neural network, assuming the input is $\textrm{D}_{x}$-dimensional,
the output is $\textrm{D}_{y}$-dimensional, and there are H hidden units?

\noindent\rule{\textwidth}{0.4pt}

\noindent 
%\begin{itemize}
%\item[$\bf{\textrm{W}}_{1}$:]{$\textrm{D}_{x}\times\textrm{H}$}
%\item[$\bf{b}_{1}$:]{$\textrm{H}$}
%\item[${\bf\textrm{W}}_{2}$:]{$\textrm{H}\times\textrm{D}_{y}$}
%\item[$\bf{b}_{2}$:]{$\textrm{D}_{y}$}
%\end{itemize}
$\bf{\textrm{W}}_{1}$ must have dimensions:  $\textrm{D}_{x}\times\textrm{H}$. The bias ($\bf{b}_{1}$) for the first layer must have
dimensions $\textrm{H}$. Adding these two together, yields $(\textrm{D}_{x} + 1)\times\textrm{H}$. Proceeding to the second layer,
there must be $\textrm{H}\times\textrm{D}_{y}$ parameters associated with the weight matrix ${\bf\textrm{W}}_{2}$. The bias ($\bf{b}_{2}$)
for the second layer must have dimensions $\textrm{D}_{y}$ elements. This yields, 
\begin{equation}
(\textrm{D}_{x} + 1)\times\textrm{H} + \textrm{D}_{y}\times(\textrm{H}+1)
\end{equation}
weights, for each input vector of dimensions $\textrm{D}_{x}$.
\clearpage
\myrunninghwhead{1}{3 (word2vec)}

\myhwtitle{1}{3(a)}{Gregory King}
\bigskip

\clearpage
\myrunninghwhead{1}{4 (Sentiment Analysis)}

\myhwtitle{1}{4(a)}{Gregory King}
\bigskip
% \noindent Given two vectors $\vec{\bf A} = 4.00 \hat{\bf i} + 7.00 \hat{\bf j}$
% and $\vec{\bf B} = 5.00 \hat{\bf i} - 2.00 \hat{\bf j}$, (a) find the magnitude
% of each vector; (b) write an expression for the vector difference 
% $\vec{\bf A} - \vec{\bf B}$ using unit vectors; (c) find the magnitude and
% direction of the vector difference $\vec{\bf A} - \vec{\bf B}$; (d) In a vector
% diagram show $\vec{\bf A}$, $\vec{\bf B}$, and $\vec{\bf A} -\vec{\bf B}$, and
% also show that your diagram agrees qualitatively with your answer in part (c).

% \noindent\rule{\textwidth}{0.4pt}

% \noindent {\bf (a)} Starting with $\vec{\bf A}$; one can see that, $A_{x} = 4.00$ and 
% $A_{y} = 7.00$, so the magnitude of $\vec{\bf A}$, commonly denoted $|\vec{\bf A}|$:\\
% \begin{eqnarray}
% |\vec{\bf A}| & = & \sqrt{A^{2}_{x} + A^{2}_{y}} = \sqrt{(4.00)^{2} + (7.00)^{2}} \\
%               & = & \sqrt{16.00 + 49.00} = \sqrt{65.00} \\
%               & = & 8.06
% \end{eqnarray}

% \noindent Similarly for $\vec{\bf B}$; $B_{x} = 5.00$ and $B_{y} = -2.00$, this means,
% the magnitude of $|\vec{\bf B}|$:
% \begin{eqnarray}
% |\vec{\bf B}| & = & \sqrt{B^{2}_{x} + B^{2}_{y}} = \sqrt{(5.00)^{2} + (-2.00)^{2}} \\
%               & = & \sqrt{25.00 + 4.00} = \sqrt{29.00}\\
%               & = & 5.39
% \end{eqnarray}
% \noindent {\bf Important check:} the magnitude of any vector is always greater
% than or equal to its largest component.

% \noindent {\bf (b)}
% \begin{eqnarray}
% \vec{\bf A} - \vec{\bf B} &=& (4.00\hat{\bf i}+7.00\hat{\bf j}) - (5.00\hat{\bf i}-2.00\hat{\bf j}) = (4.00-5.00)\hat{\bf i} + (7.00+2.00)\hat{\bf j} \\
%                           &=& (-1.00)\hat{\bf i} + (9.00)\hat{\bf j}
% \end{eqnarray}

% \noindent {\bf (c)} from (b), we know that $\vec{\bf C}=\vec{\bf A}-\vec{\bf B}= (-1.00)\hat{\bf i} + (9.00)\hat{\bf j}$; 
% proceeding in the same way as part (a) to get the magnitude:
% \begin{eqnarray}
% |\vec{\bf C}| & = & \sqrt{C^{2}_{x} + C^{2}_{y}} = \sqrt{(-1.00)^{2} + (9.00)^{2}} \\
%               & = & \sqrt{1.00 + 81.00} = \sqrt{82.00}\\
%               & = & 9.06
% \end{eqnarray}
% In order to get direction of $\vec{\bf C}$, calculate the angle $\theta$, using
% the relation, $\tan\theta = \frac{C_{y}}{C_{x}}$,
% \begin{eqnarray}
% \tan\theta & = & \frac{C_{y}}{C_{x}} = \frac{9.00}{-1.00}\\
%            & = & -9.00
% \end{eqnarray}
% and $\theta = -83.6^{\circ} + 180^{\circ} = 96.3^{\circ}$, the addition of 
% $180^{\circ}$ is due to the fact that ${\bf \arctan}$ or $\tan^{-1}$ only maps to $(-90^{\circ},90^{\circ})$,
% however $\vec{\bf C}$ is located in the second quadrant (see Figure~\ref{solution figure:question 1.42}))\\
% \begin{figure}[!h!b]
% \begin{center}
% \begin{tikzpicture}
%   \draw[step=1.cm,gray,very thin](-8.,-4.) grid (8., 9.);
%   \draw[black,thick] (-8.,0) -- (8.,0) coordinate (x axis);
%   \draw[black,thick] (0,-4.) -- (0,9.) coordinate (y axis);
%   \draw[->,black,very thick] (0,0) -- node[right=2pt] {$\vec{A}$} (4., 7.);
%   \draw[->,black,very thick] (0,0) -- node[above=2pt] {$\vec{B}$} (5., -2.);
%   \draw[->,black,dashed, very thick] (4.,7.) -- node[near end, right=6pt] {$-\vec{B}$ (from the tip of $\vec{A}$)} (-1., 9.);
%   \draw[->,black,very thick] (0,0) -- node[left=2pt] {$\vec{C} = \vec{A} - \vec{B}$} (-1., 9.);
% \end{tikzpicture}
% \end{center}\caption{Vector Diagram for (1.42 (d))\label{solution figure:question 1.42}}
% \end{figure}


% %%%%% Begin Solution %%%%%%%%%%%%%%%%%%%%%%%%%%%%%%%%%%%%%%%%%%%%%%%%%%%

% \clearpage
% \pagestyle{myheadings}
% \myrunninghwhead{1}{1.43}

% \myhwtitle{1}{1.43}{Greg King}

% \bigskip

% \noindent Write each vector in Figure~\ref{problem figure:question 1.43} in
% terms of unit vectors $\bf \hat{i}$ and $\bf \hat{j}$. (b) Use unit vectors to
% express the vector $\vec{\bf C}$, where $\vec{\bf C}=3.00\vec{\bf A} - 4.00\vec{\bf B}$
% (c) Find the magnitude and direction of $\vec{\bf C}$.
% \begin{figure}[!h!b]
% \begin{center}
% \begin{tikzpicture}
%   \draw[step=1.cm,gray,very thin](-4.,-4.) grid (4., 4.);
%   \draw[black,thick] (-4.,0) -- (4.,0) coordinate (x axis);
%   \coordinate (Or) at (0,0);
%   \coordinate (A) at ($(Or) + (70:3.6)$);
%   \coordinate (B) at ($(Or) + (210:2.4)$);
%   \coordinate (NegX) at (-1.,0);
%   \coordinate (PosX) at (3.6,0);
%   \draw[->,very thick,black](Or) -- node[left=2pt]{$\vec{\bf A}$}(A);
%   \draw[->,very thick,black](Or) -- node[below=2pt]{$\vec{\bf B}$}(B);
%   \tkzMarkAngle[fill=orange,opacity=.4](PosX,Or,A);
%   \tkzLabelAngle[pos=0.7,font=\small](PosX,Or,A){$70^{\circ}$};
%   \tkzMarkAngle[fill=orange,opacity=.4](NegX,Or,B);
% %  \tkzLabelAngle[pos=0.7,left=30pt,down=10pt,font=\small](B,Or,NegX){$30^{\circ}$};
%   \draw (NegX) node[below,font=\small] {$30^{\circ}$};
  
% \end{tikzpicture}
% \end{center}\caption{Vector Diagram for Problem (1.43)\label{problem figure:question 1.43}}
% \end{figure}

% \noindent\rule{\textwidth}{0.4pt}

% \noindent (a) We can use trigonometry to find the components of each vector.
% \begin{eqnarray}
% \vec{\bf A}   & = & (3.60~\textrm{m})\times(\cos{70^{\circ}}\hat{\bf i} + \sin{70^{\circ}}\hat{\bf j})\\
%               & = & (1.23~\textrm{m})\hat{\bf i} + (3.38~\textrm{m})\hat{\bf j}\label{1.43 vec a}
% \end{eqnarray}
% and
% \begin{eqnarray}
% \vec{\bf B}   & = & (-2.40~\textrm{m})\times(\cos{30^{\circ}}\hat{\bf i} + \sin{30^{\circ}}\hat{\bf j})\\
%               & = & (-2.08~\textrm{m})\hat{\bf i} + (-1.20~\textrm{m})\hat{\bf j}\label{1.43 vec b}
% \end{eqnarray}
% \\
% \noindent (b) Starting with the definition of $\vec{\bf C} = 3.00\vec{\bf A} - 4.00\vec{\bf B}$ and the results
% from part (a) [\ref{1.43 vec a} and \ref{1.43 vec b}], we can calculate it immediately,
% \begin{eqnarray}
% \vec{\bf C} & = & 3.00\vec{\bf A} - 4.00\vec{\bf B} \\
%             & = & 3.00\times((1.23~\textrm{m})\hat{\bf i} + (3.38~\textrm{m})\hat{\bf j}) - 4.00\times((-2.08~\textrm{m})\hat{\bf i} + (-1.20~\textrm{m})\hat{\bf j})\\
%             & = & (12.01~\textrm{m})\hat{\bf i} + (14.49~\textrm{m})\hat{\bf j}
% \end{eqnarray}
% The magnitude and direction can be found as follows (and noting that $\vec{\bf C}$ falls in the first quadrant):
% \begin{equation}
% |\vec{\bf C}| = \sqrt{(12.01~\textrm{m})^{2} + (14.49~\textrm{m})} = 19.17~\textrm{m}
% \end{equation}
% and
% \begin{equation}
% \arctan{\bigg(\frac{14.49~\textrm{m}}{12.01~\textrm{m}}\bigg)} = 51.2^{\circ}
% \end{equation}

%%%% End solution and document %%%%%%%%%%%%%%%%%%%%%%%%%%%%%%%%%%%%
\end{document}

