% Template LaTeX source file for homework problem solutions.
% Alan T. Sherman (9/9/98)
% Updated: Greg King (2014)

% Running LaTeX
%
% Name this file FOO.tex
% latex FOO
% latex FOO   
%    (You have to run latex twice to get the cross references correct.
%     Running latex creates a file FOO.dvi 
%     You can view dvi files with the program xdvi )
% xdvi FOO.dvi &
%
% lpr -d FOO.dvi
%    (To print the dvi file.   Be sure to use the "-d" print option,
%     and be sure your printer can handle dvi files (not all printers can).
%     Do NOT print with "lpr FOO.dvi", which will print tens of pages
%     of unreadable dvi source code. Printing a postscript (ps) file
%     is usually more reliable, as explained below.)
%
% dvips FOO.dvi
%    (To create a postscript file named FOO.ps 
%     which you can view with the program ghostview )
% ghostview FOO.ps &
% lpr FOO.ps
%    (To print the ps file.)

%%%%%%%%%%%%%%%%%%%%%%%%%%%%%%%%%%%%%%%%%%%%%%%%%%%%%%%%%%%%%%%%%%%%%%

\documentclass[letter,12pt]{article}

\RequirePackage{amsmath}
\RequirePackage{amsmath,amssymb,amsthm}
\RequirePackage{tikz}
\usepackage{listings}
\usepackage{color}
\usepackage{textcomp}
\usepackage{graphicx}

\renewcommand{\lstlistlistingname}{Code Listings} 
\renewcommand{\lstlistingname}{Code Listing} 
\definecolor{gray}{gray}{0.5} 
\definecolor{key}{rgb}{0,0.5,0} 
\lstnewenvironment{python}[1][]{ 
\lstset{
language=python,
basicstyle=\ttfamily\small,
otherkeywords={1, 2, 3, 4, 5, 6, 7, 8 ,9 , 0, -, =, +, [, ], (, ), \{, \}, :, *, !},
keywordstyle=\color{blue},
stringstyle=\color{red},
showstringspaces=false,
emph={class, pass, in, for, while, if, is, elif, else, not, and, or,
def, print, exec, break, continue, return},
emphstyle=\color{black}\bfseries,
emph={[2]True, False, None, self},
emphstyle=[2]\color{key},
emph={[3]from, import, as},
emphstyle=[3]\color{blue},
upquote=true,
morecomment=[s]{"""}{"""},
commentstyle=\color{gray}\slshape,
frame=tb,
rulesepcolor=\color{blue},#1
}}{}


\usetikzlibrary{calc}
\RequirePackage{tkz-euclide}
\usetkzobj{all}

\setlength{\textheight}{8.5in}
\setlength{\headheight}{.25in}
\setlength{\headsep}{.25in}
\setlength{\topmargin}{0in}
\setlength{\textwidth}{6.5in}
\setlength{\oddsidemargin}{0in}
\setlength{\evensidemargin}{0in}

\newcommand{\myN}{\hbox{N\hspace*{-.9em}I\hspace*{.4em}}}
\newcommand{\myZ}{\hbox{Z}^+}
\newcommand{\myR}{\hbox{R}}

\newcommand{\myfunction}[3]
{${#1} : {#2} \rightarrow {#3}$ }

\newcommand{\myzrfunction}[1]
{\myfunction{#1}{{\myZ}}{{\myR}}}


% Formating Macros
%

\newcommand{\myheader}[4]
{\vspace*{-0.5in}
\noindent
{#1} \hfill {#3}

\noindent
{#2} \hfill {#4}

\noindent
\rule[8pt]{\textwidth}{1pt}

\vspace{1ex} 
}  % end \myheader 

\newcommand{\myalgsheader}[0]
{\myheader{Stanford University, Department of Computer Science}
{Computer Science 224D}{Spring 2016}{Section 1}}

% Running head (goes at top of each page, beginning with page 2.
% Must precede by \pagestyle{myheadings}.
\newcommand{\myrunninghead}[2]
{\markright{{\it {#1}, {#2}}}}

\newcommand{\myrunningalgshead}[2]
{\myrunninghead{Computer Science 224D}{{#1}}}

\newcommand{\myrunninghwhead}[2]
{\myrunningalgshead{Solution to Assignment {#1}, Problem {#2}}}

\newcommand{\mytitle}[1]
{\begin{center}
{\large {\bf {#1}}}
\end{center}}

\newcommand{\myhwtitle}[3]
{\begin{center}
{\large {\bf Solution to Assignment {#1}, Problem {#2}}}\\
\medskip 
{\it {#3}} % Name goes here
\end{center}}

\newcommand{\mysection}[1]
{\noindent {\bf {#1}}}

%%%%%% Begin document with header and title %%%%%%%%%%%%%%%%%%%%%%%%%
\begin{document}

\myalgsheader

\pagestyle{plain}
\setcounter{page}{1}
\myhwtitle{1}{1 {\bf Tensorflow Softmax} (a)}{Gregory King}

\bigskip

\noindent In this question, we will implement a linear classifier with loss function
\begin{equation}
J_{\texttt{softmax-CE}}({\textbf{W}}) = CE({\boldsymbol y}, \texttt{softmax}({\boldsymbol x}\textbf{W}))
\end{equation}
Here the rows of ${\boldsymbol x}$ are feature vectors. We will use \texttt{tensorflow}'s automatic
differentiation capability to fit this model to provided data.\\

\noindent (a) Implement the softmax function using \texttt{tensorflow} in \texttt{q1\_softmax.py}. Remember that,
\begin{equation}
\texttt{softmax}({\boldsymbol x})_{i} = \frac{e^{{\boldsymbol x}_{i}}}{\sum_{j} e^{{\boldsymbol x}_j}}
\end{equation}
\textbf{Note}: that you may \textbf{NOT} use \texttt{tf.nn.softmax} or related build-in functions. You can run basic (non-exhaustive tests) by running: \texttt{python q1\_softmax.py}.\\
\vspace{5mm}
\noindent\rule{\textwidth}{0.4pt}\vspace{5mm}

\noindent Implementation in \texttt{numpy}:
\begin{python}
import numpy as np

def softmax(x):
    c = np.max(x, axis=x.ndim - 1, keepdims=True)
    #for numerical stability
    y = np.sum(np.exp(x - c), axis=x.ndim - 1, keepdims=True)
    x = np.exp(x - c)/y
    return x
\end{python}
The implementation in \texttt{tensorflow}:
\begin{python}
import numpy as np
import tensorflow as tf

def softmax(x):
  """
  Compute the softmax function in tensorflow.

  You might find the tensorflow functions tf.exp, 
  tf.reduce_max, tf.reduce_sum, tf.expand_dims useful. (Many
  solutions are possible, so you may not need to use all of
  these functions). Recall also that many common tensorflow
  operations are sugared (e.g. x * y does a tensor
  multiplication if x and y are both tensors). Make sure to
  implement the numerical stability fixes as in the previous
  homework!

  Args:
    x:   tf.Tensor with shape (n_samples, n_features). Note
         feature vectors are represented by row-vectors. (For
         simplicity, no need to handle 1-d input as in the 
         previous homework)
  Returns:
    out: tf.Tensor with shape (n_sample, n_features). You
         need to construct this tensor in this problem.
  """

  ### YOUR CODE HERE
  log_c = tf.reduce_max(x,
                        reduction_indices=[len(x.get_shape()) - 1],
                        keep_dims=True)
  y     = tf.reduce_sum(tf.exp(x - log_c),
                        reduction_indices=[len(x.get_shape()) - 1],
                        keep_dims=True)
  out   = tf.exp(x - log_c) / y
  ### END YOUR CODE
  
  return out 
\end{python}

\clearpage
\pagestyle{myheadings}
\myrunninghwhead{1}{1 (Tensorflow Softmax) (b)}

\myhwtitle{1}{1(b)}{Gregory King}

\bigskip

\noindent Implement the cross-entropy loss using \texttt{tensorflow} in \texttt{q1\_softmax.py}. Remember that:
\begin{equation}
CE({\boldsymbol y}, \hat{\boldsymbol y}) = - \sum^{\textrm{N}_{c}}_{i=1} y_{i}\log(\hat{y}_{i})
\end{equation}
where ${\boldsymbol y}\in\mathbb{R}^{5}$ is a one-hot label vector and $\textrm{N}_{c}$ is the number of classes. \textbf{Note}: that you may \textbf{NOT} use \texttt{tensorflow}'s built-in cross-entropy functions for this question. You can run basic (non-exhaustive tests) by running \texttt{python q1\_softmax.py}.

\vspace{5mm}
\noindent\rule{\textwidth}{0.4pt}
\begin{python}
import numpy as np

\end{python}
\clearpage

\myrunninghwhead{1}{1 (Tensorflow Softmax) (c)}

\myhwtitle{1}{1(c)}{Gregory King}

\bigskip

\noindent Carefully stude the \texttt{Model} class in \texttt{model.py}. Briefly explain the purpose of placeholder variables and feed dictionaries in \texttt{tensorflow} computations. Fill in the implementation for the \texttt{add\_placeholders}, \texttt{create\_feed\_dict} in \texttt{q1\_classifier.py}.

\noindent \textbf{Hint}: that configuration variables are stored in \texttt{Config} class. You will need to use these configuartion variables in the code.
\vspace{5mm}
\noindent\rule{\textwidth}{0.4pt}
\begin{python}
import numpy as np

def softmax(x):
    c = np.max(x, axis=x.ndim - 1, keepdims=True)
    #for numerical stability
    y = np.sum(np.exp(x - c), axis=x.ndim - 1, keepdims=True)
    x = np.exp(x - c)/y
    return x
\end{python}
\clearpage
\myrunninghwhead{1}{1 (Tensorflow Softmax) (d)}

\myhwtitle{1}{1(d)}{Gregory King}

\bigskip

\noindent Implement the transformatoin for a softmax classifier in function \texttt{add\_model} in \texttt{q1\_classifier.py}. Add cross-entropy loss in function \texttt{add\_loss\_op} in the same file. Use the implementation form earlier pars of the problem, \textbf{NOT} \texttt{tensorflow} built-ins.

\vspace{5mm}
\noindent\rule{\textwidth}{0.4pt}
\begin{python}
import numpy as np
\end{python}
\clearpage
\myrunninghwhead{1}{1 (Tensorflow Softmax) (e)}

\myhwtitle{1}{1(e)}{Gregory King}

\bigskip

\noindent Fill in the implementation for \texttt{add\_training\_op} in \texttt{q1\_classifier.py}. Explain how \texttt{tensorflow}'s automatic differentiation removes the need for us to define gradients explicitly. Verify that your model is able to fit to synthetic data by running \texttt{python q1\_classifier.py} and make sure that the tests pass.

\noindent \textbf{Hint}: Make sure to use the learning rate specified in \texttt{Config}.

\vspace{5mm}
\noindent\rule{\textwidth}{0.4pt}
\begin{python}
import numpy as np

\end{python}
\clearpage
\end{document}




