% Template LaTeX source file for homework problem solutions.
% Alan T. Sherman (9/9/98)
% Updated: Greg King (2014)

% Running LaTeX
%
% Name this file FOO.tex
% latex FOO
% latex FOO   
%    (You have to run latex twice to get the cross references correct.
%     Running latex creates a file FOO.dvi 
%     You can view dvi files with the program xdvi )
% xdvi FOO.dvi &
%
% lpr -d FOO.dvi
%    (To print the dvi file.   Be sure to use the "-d" print option,
%     and be sure your printer can handle dvi files (not all printers can).
%     Do NOT print with "lpr FOO.dvi", which will print tens of pages
%     of unreadable dvi source code. Printing a postscript (ps) file
%     is usually more reliable, as explained below.)
%
% dvips FOO.dvi
%    (To create a postscript file named FOO.ps 
%     which you can view with the program ghostview )
% ghostview FOO.ps &
% lpr FOO.ps
%    (To print the ps file.)

%%%%%%%%%%%%%%%%%%%%%%%%%%%%%%%%%%%%%%%%%%%%%%%%%%%%%%%%%%%%%%%%%%%%%%

\documentclass[letter,12pt]{article}

\RequirePackage{amsmath}
\RequirePackage{amsmath,amssymb,amsthm}
\RequirePackage{tikz}
\usepackage{listings}
\usepackage{color}
\usepackage{textcomp}
\usepackage{graphicx}
\usepackage{hyperref}

\renewcommand{\lstlistlistingname}{Code Listings} 
\renewcommand{\lstlistingname}{Code Listing} 
\definecolor{gray}{gray}{0.5} 
\definecolor{key}{rgb}{0,0.5,0} 
\lstnewenvironment{python}[1][]{ 
\lstset{
language=python,
basicstyle=\ttfamily\small,
otherkeywords={1, 2, 3, 4, 5, 6, 7, 8 ,9 , 0, -, =, +, [, ], (, ), \{, \}, :, *, !},
keywordstyle=\color{blue},
stringstyle=\color{red},
showstringspaces=false,
emph={class, pass, in, for, while, if, is, elif, else, not, and, or,
def, print, exec, break, continue, return},
emphstyle=\color{black}\bfseries,
emph={[2]True, False, None, self},
emphstyle=[2]\color{key},
emph={[3]from, import, as},
emphstyle=[3]\color{blue},
upquote=true,
morecomment=[s]{"""}{"""},
commentstyle=\color{gray}\slshape,
frame=tb,
rulesepcolor=\color{blue},#1
}}{}


\usetikzlibrary{calc}
\RequirePackage{tkz-euclide}
\usetkzobj{all}

\setlength{\textheight}{8.5in}
\setlength{\headheight}{.25in}
\setlength{\headsep}{.25in}
\setlength{\topmargin}{0in}
\setlength{\textwidth}{6.5in}
\setlength{\oddsidemargin}{0in}
\setlength{\evensidemargin}{0in}

\newcommand{\myN}{\hbox{N\hspace*{-.9em}I\hspace*{.4em}}}
\newcommand{\myZ}{\hbox{Z}^+}
\newcommand{\myR}{\hbox{R}}

\newcommand{\myfunction}[3]
{${#1} : {#2} \rightarrow {#3}$ }

\newcommand{\myzrfunction}[1]
{\myfunction{#1}{{\myZ}}{{\myR}}}


% Formating Macros
%

\newcommand{\myheader}[4]
{\vspace*{-0.5in}
\noindent
{#1} \hfill {#3}

\noindent
{#2} \hfill {#4}

\noindent
\rule[8pt]{\textwidth}{1pt}

\vspace{1ex} 
}  % end \myheader 

\newcommand{\myalgsheader}[0]
{\myheader{Stanford University, Department of Computer Science}
{Computer Science 224D}{Spring 2016}{Section 1}}

% Running head (goes at top of each page, beginning with page 2.
% Must precede by \pagestyle{myheadings}.
\newcommand{\myrunninghead}[2]
{\markright{{\it {#1}, {#2}}}}

\newcommand{\myrunningalgshead}[2]
{\myrunninghead{Computer Science 224D}{{#1}}}

\newcommand{\myrunninghwhead}[2]
{\myrunningalgshead{Solution to Assignment {#1}, Problem {#2}}}

\newcommand{\mytitle}[1]
{\begin{center}
{\large {\bf {#1}}}
\end{center}}

\newcommand{\myhwtitle}[3]
{\begin{center}
{\large {\bf Solution to Assignment {#1}, Problem {#2}}}\\
\medskip 
{\it {#3}} % Name goes here
\end{center}}

\newcommand{\mysection}[1]
{\noindent {\bf {#1}}}

%%%%%% Begin document with header and title %%%%%%%%%%%%%%%%%%%%%%%%%
\begin{document}

\myalgsheader

\pagestyle{plain}
\setcounter{page}{1}
\myhwtitle{2}{1 {\bf Tensorflow Softmax}}{Gregory King}

\bigskip

\noindent In this question, we will implement a linear classifier with loss function
\begin{equation}
J_{\texttt{softmax-CE}}({\textbf{W}}) = CE({\boldsymbol y}, \texttt{softmax}({\boldsymbol x}\textbf{W}))
\end{equation}
Here the rows of ${\boldsymbol x}$ are feature vectors. We will use \texttt{tensorflow}'s automatic
differentiation capability to fit this model to provided data.\\

\noindent (a) Implement the softmax function using \texttt{tensorflow} in \texttt{q1\_softmax.py}. Remember that,
\begin{equation}
\texttt{softmax}({\boldsymbol x})_{i} = \frac{e^{{\boldsymbol x}_{i}}}{\sum_{j} e^{{\boldsymbol x}_j}}
\end{equation}
\textbf{Note}: that you may \textbf{NOT} use \texttt{tf.nn.softmax} or related build-in functions. You can run basic (non-exhaustive tests) by running: \texttt{python q1\_softmax.py}.\\
\vspace{5mm}
\noindent\rule{\textwidth}{0.4pt}\vspace{5mm}

\noindent Implementation in \texttt{numpy}:
\begin{python}
import numpy as np

def softmax(x):
    c = np.max(x, axis=x.ndim - 1, keepdims=True)
    #for numerical stability
    y = np.sum(np.exp(x - c), axis=x.ndim - 1, keepdims=True)
    x = np.exp(x - c)/y
    return x
\end{python}
The implementation in \texttt{tensorflow}:
\begin{python}
import numpy as np
import tensorflow as tf

def softmax(x):
  """
  Compute the softmax function in tensorflow.

  You might find the tensorflow functions tf.exp, 
  tf.reduce_max, tf.reduce_sum, tf.expand_dims useful. (Many
  solutions are possible, so you may not need to use all of
  these functions). Recall also that many common tensorflow
  operations are sugared (e.g. x * y does a tensor
  multiplication if x and y are both tensors). Make sure to
  implement the numerical stability fixes as in the previous
  homework!

  Args:
    x:   tf.Tensor with shape (n_samples, n_features). Note
         feature vectors are represented by row-vectors. (For
         simplicity, no need to handle 1-d input as in the 
         previous homework)
  Returns:
    out: tf.Tensor with shape (n_sample, n_features). You
         need to construct this tensor in this problem.
  """

  ### YOUR CODE HERE
  log_c = tf.reduce_max(x,
                        reduction_indices=[len(x.get_shape()) - 1],
                        keep_dims=True)
  y     = tf.reduce_sum(tf.exp(x - log_c),
                        reduction_indices=[len(x.get_shape()) - 1],
                        keep_dims=True)
  out   = tf.exp(x - log_c) / y
  ### END YOUR CODE
  
  return out 
\end{python}

\clearpage
\pagestyle{myheadings}
\myrunninghwhead{2}{1 (Tensorflow Softmax)}

\myhwtitle{2}{1(b)}{Gregory King}

\bigskip

\noindent Implement the cross-entropy loss using \texttt{tensorflow} in \texttt{q1\_softmax.py}. Remember that:
\begin{equation}
CE({\boldsymbol y}, \hat{\boldsymbol y}) = - \sum^{\textrm{N}_{c}}_{i=1} y_{i}\log(\hat{y}_{i})
\end{equation}
where ${\boldsymbol y}\in\mathbb{R}^{5}$ is a one-hot label vector and $\textrm{N}_{c}$ is the number of classes. \textbf{Note}: that you may \textbf{NOT} use \texttt{tensorflow}'s built-in cross-entropy functions for this question. You can run basic (non-exhaustive tests) by running \texttt{python q1\_softmax.py}.

\vspace{5mm}
\noindent\rule{\textwidth}{0.4pt}
\begin{python}
import numpy as np

\end{python}
\clearpage

\myrunninghwhead{2}{1 (Tensorflow Softmax)}

\myhwtitle{2}{1(c)}{Gregory King}

\bigskip

\noindent Carefully stude the \texttt{Model} class in \texttt{model.py}. Briefly explain the purpose of placeholder variables and feed dictionaries in \texttt{tensorflow} computations. Fill in the implementation for the \texttt{add\_placeholders}, \texttt{create\_feed\_dict} in \texttt{q1\_classifier.py}.

\noindent \textbf{Hint}: that configuration variables are stored in \texttt{Config} class. You will need to use these configuartion variables in the code.
\vspace{5mm}
\noindent\rule{\textwidth}{0.4pt}
\begin{python}
import numpy as np
\end{python}
\clearpage
\myrunninghwhead{2}{1 (Tensorflow Softmax)}

\myhwtitle{2}{1(d)}{Gregory King}

\bigskip

\noindent Implement the transformatoin for a softmax classifier in function \texttt{add\_model} in \texttt{q1\_classifier.py}. Add cross-entropy loss in function \texttt{add\_loss\_op} in the same file. Use the implementation form earlier pars of the problem, \textbf{NOT} \texttt{tensorflow} built-ins.

\vspace{5mm}
\noindent\rule{\textwidth}{0.4pt}
\begin{python}
import numpy as np
\end{python}
\clearpage
\myrunninghwhead{2}{1 (Tensorflow Softmax)}

\myhwtitle{2}{1(e)}{Gregory King}

\bigskip

\noindent Fill in the implementation for \texttt{add\_training\_op} in \texttt{q1\_classifier.py}. Explain how \texttt{tensorflow}'s automatic differentiation removes the need for us to define gradients explicitly. Verify that your model is able to fit to synthetic data by running \texttt{python q1\_classifier.py} and make sure that the tests pass.

\noindent \textbf{Hint}: Make sure to use the learning rate specified in \texttt{Config}.

\vspace{5mm}
\noindent\rule{\textwidth}{0.4pt}
\begin{python}
import numpy as np
\end{python}
\clearpage
\myrunninghwhead{2}{2 (Deep Networks for NER)}

\myhwtitle{2}{2(a)}{Gregory King}

\bigskip

\noindent In this section, we'll get to practice backpropagation and training deep networks to attack the task of \textbf{Named Entity Recognition} (\textbf{NER}): predicting whether a given word, in context, represets one of four categories:
\begin{itemize}
\item{Person (\texttt{PER})}
\item{Organization (\texttt{ORG})}
\item{Location (\texttt{LOC})}
\item{Miscellaneous (\texttt{MISC})}
\end{itemize}
\noindent We formulate this as a 5-class classification problem, using the four above classes and a null-class (\texttt{0}) for words that do not represent a named entity (\textbf{NOTE}: most words fall into this category).

\noindent The model is a 1-hidden-layer neural network, with an additional representation layer similar to what you saw with $\texttt{word2vec}$. Rather than averaging or sampling, here we explicitly represent context as a "window" consisting of a word concatenated with its immediate neighbours:

\begin{equation}
{\boldsymbol x}^{(t)} = [ {\boldsymbol x}_{t-1}\textbf{L}, {\boldsymbol x}_{t}\textbf{L}, {\boldsymbol x}_{t+1}\textbf{L}]\in\mathbb{R}^{3\textrm{d}}
\end{equation}
where the input ${\boldsymbol x}_{t-1}$, ${\boldsymbol x}_{t}$, ${\boldsymbol x}_{t+1}$ are one-hot row vectors into an embedding matrix $\textbf{L}\in\mathbb{R}^{|\textrm{V}|\times\textrm{d}}$, which each row $\textbf{L}_{i}$ as the vector for a particular word $i={\boldsymbol x}_{t}$. We then compute our prediction as:
\begin{align}
1 &=
2 &=
\end{align}
And evaluate by cross-entropy loss,
\begin{equation}
J(\theta) = CE({\boldsymbol y},\hat{\boldsymbol y}) = - \sum^{\textrm{d}}_{i=1} y_{i}\log({\hat{y}}_{i})
\end{equation}
To compute the loss for the training set, we sum (or average) this $J(\theta)$ as computed with respect to each training example.

\noindent For this problem, we let $\textrm{d} = 50$ be the length of our word vectors, which are concatenated into a window of width $3\times5=150$. The hidden layer has dimensions of 100, and the output layer $\hat{y}$ has dimension of 5.

\noindent (a) Compute the gradients of $J(\theta)$ with respect to all the model parameters:
\begin{equation}
\frac{\partial J}{\partial{\boldsymbol U}}\quad\frac{\partial J}{\partial{\boldsymbol b}_{2}}\quad\frac{\partial J}{\partial{\boldsymbol W}}\quad\frac{\partial J}{\partial{\boldsymbol b}_{1}}\quad\frac{\partial J}{\partial{\boldsymbol L}_{i}}
\end{equation}
where,
\begin{equation}
{\boldsymbol U}\in\mathbb{R}^{100\times5}\quad{\boldsymbol b}_{2}\in\mathbb{R}^{5}\quad{\boldsymbol W}\in\mathbb{R}^{150\times100}\quad{\boldsymbol b}_{1}\in\mathbb{R}^{100}\quad{\boldsymbol L}_{i}\in\mathbb{R}^{50}
\end{equation}

\noindent In the spirit of backpropagation, you should express the derivative of activation functions ($\tanh,\texttt{softmax}$) in terms of their function values (as with sigmoid in Assignment~1). This identity may be helpful:
\begin{equation}
\tanh(z) = 2\sigma(2z)-1
\end{equation}

\noindent Furthermore, you should express the gradients by using an "error vector" propagated back to each layer; this just amounts to putting parentheses around factors in the chain rule, and will greatly simplify your analysis. All resulting gradients should have simple, closed-form expressions in terms of matrix operations. (\textbf{Hint}: you've already done most of the work here as part of Assignment~1.)

\vspace{5mm}
\noindent\rule{\textwidth}{0.4pt}

\clearpage
\myrunninghwhead{2}{2 (Deep Networks for NER)}

\myhwtitle{2}{2(b)}{Gregory King}

\bigskip

\noindent To avoid parameters from exploding or becoming highly correlated, it is helpful to augment our cost function with a Gaussian prior: this tends to push parameter weights closer to zero, without constraining their direction, and often leads to classifiers with better generalization ability.

\noindent If we maximize \textbf{log-likelihood} (as with the cross-entropy loss above), then the Gaussian prior becomes a quadratic term\footnote{Optional (\textbf{not graded}): The interested reader should prove that this is indeed the maximum-likelihood objective when we let ${\boldsymbol W}_{ij} \equiv \textrm{N}(0, 1/\lambda)$ for all $i$, $j$.} (L2 regularization):
\begin{equation}
J_{\textrm{reg}}(\theta) = \frac{\lambda}{2}\big[\sum_{i,j}{\boldsymbol W}^{2}_{ij} + \sum_{i\prime,j\prime}{\boldsymbol U}^{2}_{i\prime j\prime} \big]
\end{equation}

\noindent Update your gradients from part (a) to include the additional term in this loss function (i.e. compute $\frac{\partial J_{\textrm{full}}}{\partial{\boldsymbol W}}$, etc.).
\vspace{5mm}
\noindent\rule{\textwidth}{0.4pt}

\clearpage

\myrunninghwhead{2}{2 (Deep Networks for NER)}

\myhwtitle{2}{2(c)}{Gregory King}

\bigskip

\noindent In order to avoid neurons becoming too correlated and ending up in poor local minima, it is often helpful to randomly initialize parameters. One of the most frequent initializations used is called \textbf{Xavier} initialization\footnote{This is also referred to as \textbf{Glorot} initialization and was initially described in \url{http://jmlr.org/proceedings/papers/v9/glorot10a/glorot10a.pdf}~.}.

\noindent Given a matrix ${\boldsymbol A}$ of dimensions $\textrm{m}\times\textrm{n}$, select values ${\boldsymbol A}_{ij}$ uniformly from $[-\epsilon,-\epsilon]$, where
\begin{equation}
\epsilon = \frac{\sqrt{6}}{\sqrt{m+n}}
\end{equation}
Implement the initialization for use in \texttt{xavier\_weight\_init} in \texttt{q2\_initialization.py} and use it for the weights ${\boldsymbol W}$ and ${\boldsymbol U}$.
\vspace{5mm}
\noindent\rule{\textwidth}{0.4pt}

\clearpage

\end{document}
